The area of science that is complex systems have grown rapidly over the past
years with increasing interest and areas of application. One part that is very
frequently brought up is evolution and the behavior of animals. Over the course
of history, humans have always been fascinated by and also able to learn a lot
from the animal kingdom in terms of why they look like they do and why they
behave like they do. They are all there simply because they survived and nature
found many solutions for how each animal eventually adapted to its environment
and therefore could persist. Some animals in the air as well as in the water
found the behavior called swarming, a phenomena that is an ideal example of a
complex system since knowing the behavior of one individual one can not
determine the behavior of the group of individuals. One benefit with the
swarming behavior is the ability to confuse and avoid predators, which for
example can commonly be seen with sharks hunting fish in shoals.\\
\\
There has been quite a significant amount of research made on the swarming
phenomena regarding predator avoidance and many algorithms to model this
behavior exists. This has however mostly been done with a predator being
programmed to catch prey in a certain way, but the predator in question has,
through evolution as well, learned how to overcome the swarming benefits to be
able to catch prey. The goal with this project is to investigate if it is
possible to by using an artificial neural network, model and train a predator
to catch the prey out of a given swarming model.
