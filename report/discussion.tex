The model is found to show promising results in developing an evolving predatory
behaviour in the corresponding manner to many evolving prey behaviour
simulations. The shark showed great improvement in hunting technique when
allowed to evolve via an artifical neural network.\\
\\
The comparison of the neural network and programmed AI shows that the
programmed AI yields a fairly static result, meaning that it might be
of greater value when training a neural network fish shoal, at least
initially. The fluctuations in the neural networks fitness values point to
that there is still a large amount of stochasticity governing the model. If
this was further reduced it is probable that the neural network shark could
vastly outperform a programmed AI. The usability of this remains questionable
in a general sense, and it is believed that the amount of time and effort
required to achieve the neural network training of the shark is only appliable
when competing two neural networks against each other.\\
\\
The key factor that promotes the usage of artificial neural networks in
predatory behaviour development is using this in combination with an
evolutionary prey behaviour in order to model a natural behavioural evolution
between predator and prey.
