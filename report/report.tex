\documentclass[12pt,A4]{article}
\usepackage[utf8]{inputenc}
\usepackage{graphicx}
\usepackage{psfrag}
\usepackage{dcolumn}
\usepackage{bm}
\usepackage{amsmath}
\usepackage{amssymb}
\usepackage{latexsym}
\usepackage{subcaption}
\addtolength{\topmargin}{-1.9cm}
\addtolength{\textheight}{5.5cm}
\addtolength{\evensidemargin}{-1.2cm}
\addtolength{\oddsidemargin}{-1.2cm}
\addtolength{\textwidth}{2cm}
\title{Usage of neural networks to model marine predatory behaviour.}
\author{Henrik Adolfsson, Andreas Magnusson, Kristian Onsj\"o}
\date{January 10 2016}

\begin{document}
\parindent=0cm

\maketitle

\begin{abstract}
Abstract goes here.
\end{abstract}

\section{Introduction}
The area of science that is complex systems have grown rapidly over the past years with increasing interest and areas of application. One part that is very frequently brought up is evolution and animal behaviour. Over the course of history, humans have always been fascinated by and also able to learn a lot from how animals do things, why they look like they do and why they behave like they do. They are all there simply because they survived but nature found many solutions for how each animal eventually adapted and could persist. Some animals in the air as well as in the water found swarming, a phenomena that is an ideal example of a complex system; knowing the behavior of one individual, one can not determine the behavior of the group. One benefit with the swarming behavior is the ability to confuse and avoid predators, meaning that the predators in question had to somehow learn to overcome this to survive themselves.

\section{Method}

\section{Result}

\section{Discussion}





\end{document}

