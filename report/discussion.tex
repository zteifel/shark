The model is found to show promising results in developing an evolving predatory
behavior in the corresponding manner to many evolving prey behavior
simulations. The shark showed great improvement in hunting technique when
allowed to evolve via an artificial neural network, so in the matter of the question of whether it is possible to model a predator this way, the answer is yes. Is it better than a simple programmed AI?\\
\\
The comparison of the neural network and programmed AI shows that the
programmed AI yields a better shark behavior pretty much all the time. By making interactive plots and looking at an ANN shark and an AI shark hunting fish, one can see that the behaviors differs quite significantly. The AI can overcome the problem of having limited turning radius compared to speed and this proves to be a key part in catching fish more efficiently. The ANN struggles on this part and it seems as if the behavior pattern indicates the neural network might not be sufficiently large to be able to learn movements that has such big and quick turns. Further investigation of this was done by training the ANN on two very similar parameter settings where the ones of importance are the shark max speed put to 5, the fish drift speed at 3 and the fish max speed at 4 and then 5. With fish max speed 4 one can see that the ANN tends towards the same behavior as the AI but moving slower, the slow speed however making it possible to make the narrow turns. With speed max speed of 5 the ANN moves at max speed all the time just making circles of strikes repeatedly and can not make the narrow turns anymore. This indicates even more what is stated above, the ANN is too "slow" i.e. the brain is to small.
Widening the discussion of parameters that could potentially affect the results of this model could make a longer discussion than is within scope of this report (a table of all known potential variables can be seen in the appendix). With a project of this kind, one must carefully consider the balance between realism of the model and the limited tools and time to finish it. A sample of our first discussion in the group regarding sharks and fish consisted of questions such as: How do sharks see? How do fish see? Should the shark have a time limit, such as max stamina or energy? How fast can sharks move compared to fish? Sharks need to move to be able to breath, should we let it die if it stands still too long or just let it move constantly? What sort of level of calculations can our computers handle? How many fishes in the shoal?
Other things that was equally important to consider was the parameters and structure of the ANN and the training algorithm and which parameters to use there. Since what is mentioned above is a rather short version of all the matters brought up in the discussion, one can understand that this problem is very complicated. With the lack of time to investigate how changes in each of the parameters would affect the model, we had to go with a lot of guessing instead. This can be close to stabbing in the dark some times since some variables might have a huge effect where intuitively it would not. An example of this was in the early stages of the project we were trying to get an increasing fitness trend by calculating fitness only based on the shark's distance from the fish shoal and narrowing the creep rate in the mutation step of the GA, yielded a significant improvement.
If more time would have been given, our next step would be to make more runs to compare things such as will 5 runs with 1000 max energy give the same result as one run on 5000 max energy. From there what seems like a good next step is to increase the number of layers in the ANN to see what effect it has. The energy of the shark would be considered more adding more and more detail, such as turning would make it lose energy faster and so on. 
The key factor that promotes the usage of artificial neural networks in
predatory behavior development is using this in combination with an
evolutionary prey behavior in order to model a natural behavioral evolution
between predator and prey.
